\documentclass[9pt,twocolumn,twoside,lineno]{pnas-new}

%% Some pieces required from the pandoc template
\providecommand{\tightlist}{%
  \setlength{\itemsep}{0pt}\setlength{\parskip}{0pt}}

% Use the lineno option to display guide line numbers if required.
% Note that the use of elements such as single-column equations
% may affect the guide line number alignment.


\usepackage[T1]{fontenc}
\usepackage[utf8]{inputenc}



\templatetype{pnasresearcharticle}  % Choose template

\title{A Standardized Effect Size for Evaluating and Comparing the Strength of
Phylogenetic Signal}

\author[a,2]{Dean C. Adams}
\author[a,b]{Erica K. Baken}
\author[b]{Michael L. Collyer}

  \affil[a]{Department of Ecology, Evolution, and Organismal Biology, Iowa State
University, Ames, Iowa, 50010. USA.}
  \affil[b]{Department of Science, Chatham University, Pittsburgh, Pennsylvania,
15232. USA.}


% Please give the surname of the lead author for the running footer
\leadauthor{Adams et al.}

% Please add here a significance statement to explain the relevance of your work
\significancestatement{Evolutionary biologists wish to quantify and compare the strength of
phylogenetic signal across datasets, but analytical tools for these
comparisons are generally lacking. Here we develop a standardized effect
size, \(Z_K\), which measures the strength of phylogenetic signal on a
common statistical scale. We also provide a test statistic,
\(\hat{Z}_{12}\), for comparing the strength of phylogenetic signal
across datasets. We find that two commonly used parameters (Pagel's
\(\lambda\) and Blomberg's \emph{K}), not converted to effect sizes, are
unsuitable for this purpose. Our effect-size procedure enables
biologists to quantitatively address hypotheses that compare the
strength of phylogenetic signal between various phenotypic traits, even
when those traits are found in different evolutionary lineages or have
different units or scales.}


\authorcontributions{D.C.A. designed the research; D.C.A., E.K.B., and M.L.C. performed the
research and wrote the paper.}

\authordeclaration{The authors declare no conflict of interest. \hfill\break

Data deposition: Data for the empirical example may be found on DRYAD:
\url{doi:10.5061/dryad.b554m44} and \url{doi:10.5061/dryad.59zw3r23m}.
R-scripts for simulation tests are found on Github: XXX. Computer code
for implementing the two-sample comparison of effect sizes is found in
geomorph:
\url{https://cran.r-project.org/web/packages/geomorph/index.html}
\hfill\break

\textsuperscript{2}To whom correspondence should be addressed. E-mail:
dcadams@iastate.edu}


\correspondingauthor{\textsuperscript{} }

% Keywords are not mandatory, but authors are strongly encouraged to provide them. If provided, please include two to five keywords, separated by the pipe symbol, e.g:
 \keywords{  phylogenetic signal |  macroevolution |  lambda |  kappa  } 

\begin{abstract}
Macroevolutionary studies frequently characterize the phylogenetic
signal in phenotypes, however, analytical tools for comparing the
strength of that signal across traits remain largely underdeveloped.
Here we evaluate the efficacy of Pagel's \(\lambda\) to correctly
estimate the strength of phylogenetic signal in phenotypic traits across
a range of input values. We find that \(\lambda\) behaves as a Bernoulli
random variable, where estimates are increasingly skewed at larger and
smaller input levels of phylogenetic signal. Further, the precision of
\(\lambda\) varies with input signal. Another measure, Blomberg's
\emph{K}, is more consistent across a range of tree sizes, and exhibits
a positive relationship with input levels of phylogenetic signal.
However, that relationship is decidedly nonlinear. Thus, neither
\(\lambda\) nor \emph{K} are suitable as effect sizes for measuring the
strength of phylogenetic signal, and comparing that signal across
datasets. As an alternative, we propose a standardized effect size based
on \emph{K}, (\(Z_K\)), which measures the strength of phylogenetic
signal more reliably than does \(\lambda\), and places that signal on a
common scale for statistical comparison. We develop tests based on
\(Z_K\) to provide a mechanism for formally comparing the strength of
phylogenetic signal across datasets, in much the same manner as effect
sizes may be used to summarize patterns in quantitative meta-analysis.
Our approach extends the phylogenetic comparative toolkit to address
hypotheses that compare the strength of phylogenetic signal between
various phenotypic traits, even when those traits are found in different
evolutionary lineages or have different units or scales.
\end{abstract}

\dates{This manuscript was compiled on \today}
\doi{\url{www.pnas.org/cgi/doi/10.1073/pnas.XXXXXXXXXX}}

\begin{document}

% Optional adjustment to line up main text (after abstract) of first page with line numbers, when using both lineno and twocolumn options.
% You should only change this length when you've finalised the article contents.
\verticaladjustment{-2pt}

\maketitle
\thispagestyle{firststyle}
\ifthenelse{\boolean{shortarticle}}{\ifthenelse{\boolean{singlecolumn}}{\abscontentformatted}{\abscontent}}{}

% If your first paragraph (i.e. with the \dropcap) contains a list environment (quote, quotation, theorem, definition, enumerate, itemize...), the line after the list may have some extra indentation. If this is the case, add \parshape=0 to the end of the list environment.

\acknow{We thank E. Glynne and B. Juarez for comments on early drafts of the
manuscript. This work was supported in part by NSF grant DBI-1902511 (to
D.C.A.) and DBI-1902694 (to M.L.C.). \hfill\break}

Investigating macroevolutionary patterns of trait variation requires a
phylogenetic perspective, because the shared ancestry of species
violates the assumption of independence among trait values that is
common for statistical tests (1, 2). Accounting for this evolutionary
non-independence is the purview of \emph{phylogenetic comparative
methods} (PCMs): a suite of analytical tools that condition the data by
the phylogenetic relatedness of observations (3--10). PCMs are
predicated on the notion that phylogenetic signal -- the tendency for
closely related species to display similar trait values -- is present in
cross-species datasets (1, 11, 12). Indeed, under numerous evolutionary
models, phylogenetic signal is expected, as stochastic character change
along the hierarchical structure of the tree of life generates trait
covariation among taxa (1, 12, 13).

Several analytical tools have been developed to quantify phylogenetic
signal in phenotypic datasets (11, 12, 14--17), and their statistical
properties -- namely type I error rates and statistical power -- have
been investigated to determine under what conditions phylogenetic signal
can be detected (13, 16, 18--23). One of the most widely used methods
for characterizing phylogenetic signal is Pagel's \(\lambda\) (11),
which transforms the lengths of the internal branches of the phylogeny
to improve the fit of data to the phylogeny via maximum likelihood (11,
24). When incorporated in PGLS, \(\lambda\) serves as a tuning parameter
which is optimized via log-likelihood profiling while evaluating the
covariation between the dependent and independent variables, given the
phylogeny (11, 24). To infer whether phylogenetic signal differs from no
signal or a Brownian motion model of evolutionary divergence, the
observed model fit using \(\hat\lambda\) may be statistically compared
to that using \(\lambda=0\) or \(\lambda=1\) via likelihood ratio tests
(24--26) or confidence limits (27).

Another widely used measure is Blomberg's \emph{K} (12), which
characterizes phylogenetic signal as the ratio of observed trait
variation to the amount of variation expected under Brownian motion.
Blomberg's \emph{K} can be treated as a test statistic by employing a
permutation test to generate its sampling distribution (12, 16) for
determining whether significant phylogenetic signal is present in data.
Both \(\lambda\) and \emph{K} seem intuitive to interpret, as a value of
\(0\) for both corresponds to no phylogenetic signal, while a value of
\(1\) corresponds to the amount of phylogenetic signal expected under
Brownian motion. Thus, it is tempting to regard both \(\lambda\) and
\emph{K} as descriptive statistics that measure the relative strength of
phylogenetic signal, providing an estimate of its magnitude for
comparison.

The appeal of Pagel's \(\lambda\) and Blomberg's \emph{K} as descriptive
statistics is that they provide a basis for interpreting ``weak'' versus
``strong'' phylogenetic signal; i.e., small versus large values of
\(\hat{\lambda}\) or \emph{K}, respectively, in a comparative sense
(28--30). Nonetheless, an important question that has yet to be
considered is whether such comparisons are analytically appropriate, and
whether these statistics are, or can be, converted to effect sizes for
comparative analyses across datasets. To be statistics representing
phylogenetic signal, they should have reliable distributional
properties, which could be revealed with simulation experiments. For
instance, as a proportional random variable bounded by \(0\) and \(1\),
we might expect that \(\hat{\lambda}\) is a random variable that follows
the distribution of a Bernoulli probability parameter (31); i.e., branch
lengths in a tree are scaled proportionally to the probability that data
arise from a BM process. Given a known \(\lambda\) value used to
generate random data on a tree, we would also expect that the mean of an
empirical sampling distribution of \(\hat{\lambda}\) would approximately
equal \(\lambda\); the dispersion of \(\hat{\lambda}\) would be largest
at intermediate values of \(\lambda\), \(\hat{\lambda}\) would be
predictable over the range of \(\lambda\) with respect to tree size; the
distribution of \(\hat{\lambda}\) would be symmetric at intermediate
values of \(\lambda\) and more skewed toward values of 0 or 1; and that
the distribution of \(\hat{\lambda}\) would be more platykurtic at
intermediate values of \(\lambda\), becoming more leptokurtic toward 0
and 1 (31). Prior work (18) seems to support some of these conjectures,
based superficially on statistical moments for a given tree size (mean,
variance, skewness, and kurtosis; see Fig. 2 of ref. (18)). However,
because the ``strength of Brownian motion'' was simulated as a varied
weighted-average of data simulated on trees with \(\lambda=0\) and
\(\lambda=1\) and not as prescribed values of \(\lambda\) (18),
interpretation of these patterns is challenging.

By contrast, for Blomberg's \emph{K}, which is positively unbounded, we
might expect that for any \(\lambda\) used to generate data, estimates
of \emph{K} might be a random variable that follows a normal
distribution, with values distributed symmetrically (31). This attribute
seemed less reasonable based on the simulations performed by Münkemüller
et al.~(18), which suggested that distributions were positively skewed
and that Blomberg's \emph{K} might not behave as a statistic that
follows a normal distribution. However, because their simulations used a
weighted combination of simulated phylogenetic signal strengths, strong
inferences are not possible (and distributional attributes were not the
intended result of their simulations). Thus, for both Pagel's
\(\lambda\) or Blomberg's \emph{K}, evaluation of statistical moments
across a range of \(\lambda\) used to generate data would be valuable
for adjudicating the reliability of these statistics as effect sizes.
Furthermore, the expected values of these statistics appear to vary with
tree size (18), making comparisons across studies challenging.
Therefore, transformation of these statistics into \(Z\)-scores would
allow evaluation of the efficacy of each statistic to yield effect sizes
that could be used for comparisons of the strength of phylogenetic
signal across traits and lineages.

Here we use simulation experiments to compare the distributional
attributes of \(\hat{\lambda}\) and \emph{K}, plus their effect sizes
(\(Z\)-scores), across a range of tree size and phylogenetic signal
strength. We find that estimates of \(\hat{\lambda}\) are increasingly
skewed at larger and smaller input levels of phylogenetic signal and at
smaller tree sizes, vary widely for a given input value of \(\lambda\),
and that the precision of \(\hat{\lambda}\) is not constant across its
range. By contrast, estimates of \emph{K} are more consistent across
tree sizes, and are normally distributed across the range of input
levels of \(\lambda\), making \emph{K} a more reliable statistic. We
then propose an effect size based on \emph{K}, (\(Z_K\)), which provides
consistent estimates of the strength of phylogenetic signal across tree
sizes and signal strength, and facilitates quantitative comparisons of
the relative strength of phylogenetic signal across datasets.

\hypertarget{results}{%
\section{Results}\label{results}}

\begin{figure}
\centering
\includegraphics[width=8.25cm]{{fig.1}.png}
\caption{Response of Pagel's \(\lambda\) to increasing strength of
Brownian motion. Gray line signifies the 1:1 line where the input value
matches the estimate \(\hat\lambda\). At each input level, the dark
black line represents the empirically derived expected value (mean) of
\(\hat\lambda\), the red line is the standard deviation of
\(\hat\lambda\), and the blue line is Shapiro Wilks statistic of
\(\hat\lambda\) (\(W=1.0\) signifies normality, \(W< 1.0\) represent
skewed distributions).{}}
\end{figure}

\textbf{Lambda (\(\lambda\)) estimates of phylogenetic signal are
inaccurate.} Computer simulations reveal that for \(\hat{\lambda}\), the
distributional expectations of a Bernoulli variable were mostly upheld.
First, the mean value of \(\hat{\lambda}\) increases as \(\lambda\)
increases. Second, the precision in estimating \(\lambda\) varies across
the range of input values, as the standard deviation of
\(\hat{\lambda}\) is largest at intermediate values of \(\lambda\) and
smallest at extreme values (Fig.1 red line). Third, the distributions of
\(\hat{\lambda}\) tend toward normal distributions at intermediate
levels of \(\lambda\) but become increasingly skewed at more extreme
values of \(\lambda\) (Fig. 1 blue line). For small tree sizes, it is
also clear that distributions are more platykurtic at intermediate
values of \(\hat{\lambda}\). However, the mean value of
\(\hat{\lambda}\) is negatively-biased (particularly for small tree
sizes but also consistently across most of its range; Fig. 1 black line)
and standard deviations of \(\hat{\lambda}\) are negatively associated
with tree size. For tress of 128 species or less, \(\hat{\lambda}\) are
quite variable, except for cases when \(\lambda\) is near or equal to
\(1\). Taken together these results reveal that \(\hat\lambda\) is a
biased statistic that inconsistently estimates phylogenetic signal, both
across tree sizes and across the range of input values. Additional
simulations (Supporting Information) reveal that incorporating
\(\hat\lambda\) in PGLS ANOVA and regression does not adversely affect
the statistical properties of PGLS parameter estimation or model
evaluation (type I error, power, bias in coefficients). Thus, it is
reasonable to incorporate \(\hat\lambda\) in PGLS as a parameter for
tuning the degree of phylogenetic signal in the dependent variables
during the analysis. However, the statistical properties shown in Fig. 1
demonstrate that \(\lambda\) is unsuitable as an effect size for
measuring the strength of phylogenetic signal in data, and thus
\(\lambda\) should not be used for comparing phylogenetic signal across
datasets.

\begin{figure}
\centering
\includegraphics[width=8.25cm]{{fig.2}.png}
\caption{Response of Blomberg's *K* to increasing strength of Brownian motion.  At each input level, the black line represents the empirically derived expected value (mean) of
*K*, the red line is the standard deviation of
*K*, and the blue line is Shapiro Wilks statistic of
*K* (\(W=1.0\) signifies normality, \(W< 1.0\) represent
skewed distributions).{}}
\end{figure}

\textbf{Kappa (\emph{K}) estimates of phylogenetic signal are more
reliable.} Simulation results demonstrate that \emph{K} displays better
statistical properties. First, as expected, mean values of \emph{K}
increase with increasing signal (\(\lambda\)) irrespective of tree size,
albeit non-linearly (Fig. 2 black line). Second, the standard deviation
of \emph{K} is consistent across tree sizes (Fig. 2 red line), and while
it increases with \(\lambda\), it is always less than the mean (low
coefficient of variation). This finding is perhaps unsurprising, as
\emph{K} is lower-bounded by 0, and is never large for small values of
\(\lambda\). Importantly, \emph{K} is normally distributed across the
range of input \(\lambda\); a consistent pattern regardless of tree size
(Fig. 2 blue line). This differs from results of (18), where the skewing
appears to be due to combining random values generated independently,
rather than being a property of \emph{K} itself. Overall, these findings
reveal that while \emph{K} is more reliable as an estimate of
phylogenetic signal, the non-linear scaling with input signal implies
that it should not be considered an effect size that measures the
strength of phylogenetic signal on a common scale for comparison across
datasets.

\textbf{Effect sizes from \emph{K} (\(Z_K\)) better characterize
phylogenetic signal.} To measure the strength of phylogenetic signal on
a common scale, we propose effect sizes (Z-scores) for both \(\lambda\)
and \emph{K}. Statistically, a standardized effect size may be found as:

\begin{align}
    Z_{\theta}=\frac{\theta_{obs}-E(\theta)}{\sigma_\theta}
\end{align}

where \(\theta_{obs}\) is the observed test statistic, \(E(\theta)\) is
its expected value under the null hypothesis, and \(\sigma_\theta\) is
its standard error (32--34). Typically, \(\theta_{obs}\) and
\(\sigma_\theta\) are estimated from the data, while \(E(\theta)\) is
obtained from the distribution of \(\theta\) derived from parametric
theory. However, recent advances in resampling theory (35--38) have
shown that \(E(\theta)\) and \(\sigma_\theta\) may also be obtained from
an empirical sampling distribution of \(\theta\) simulated from
permutation procedures.

Formalizing the suggestion of Adams and Collyer (39), an effect size for
\emph{K} may be found as:

\begin{align}
    Z_K=\frac{K_{obs}-\hat\mu_{K}}{\hat\sigma_{K}},
\end{align}

where \(K_{obs}\) is the observed phylogenetic signal, and \(\hat\mu_K\)
and \(\hat\sigma_K\) are the mean and standard deviation of the
empirical sampling distribution of \emph{K} obtained via permutation.
The empirical sampling distribution of \emph{K} can be first transformed
via a Box-Cox transformation to better adhere to the assumption of
normality.

For \(\lambda\), deriving an effect size is more challenging, as
\(\lambda\) does not have a sampling distribution from which the
standard error and confidence intervals may be obtained, and estimates
from the Hessian matrix from PGLS are unreliable (23). Confidence
intervals are therefore generated for the values of \(\lambda\) that
intersect the log-likehihood profile for corresponding percentiles of
the \(\chi^2\) distribution used to compare the putative model to a null
model with \(\lambda = 0\) (40). Thus, an effect size for \(\lambda\)
may be found as:

\begin{align}
   \lvert Z_{\lambda} \rvert = \sqrt{\chi^2_{\hat{\lambda}}}
\end{align}

where, \(\hat{\lambda}\) is the maximized likelihood value of
\(\lambda\) and \(\chi^{2}_{\hat{\lambda}}\) is the likelihood ratio
statistic for the value.

Simulations reveal that both \(Z_{\lambda}\) and \(Z_K\) are associated
with input phylogenetic signal (\(\lambda\)), indicating that both
statistics capture the observed signal (Fig. 3). However, effect sizes
from \(\hat{\lambda}\) made little sense, as they are more strongly
associated with tree size than they are with the actual phylogenetic
signal in the data (Fig. 3). By contrast, \(Z_K\) is much more
consistent across tree sizes, and increases more linearly with
increasing levels of phylogenetic signal. Additionally, \(Z_K\) exhibits
a much stronger association with phylogenetic signal strength as
compared to tree size (Fig. 3), and its standard deviation is more
consistent, implying similar levels of precision across the range of
input signal (Supporting Information). Thus, \(Z_K\) is a more reliable
measure of the strength of phylogenetic signal, and may be used to
compare levels of phylogenetic signal across datasets.

\begin{figure}
\centering
\includegraphics[width=8.25cm]{{fig.3}.png}
\caption{Response of effect sizes \(Z_{\lambda}\) and \(Z_K\) to increasing strength of Brownian motion.  Means from simulation runs are shown for comparative ease. Individual values from each simulation run are available in Supporting Informaton. }
\end{figure}

\textbf{A test statistic (\(\hat{Z}_{12}\)) allows meaningful
comparisons across datasets.} To statistically compare the strength of
phylogenetic signal across datasets we propose a two-sample test
statistic (\(\hat{Z}_{12}\)). Based on statistical theory, a two-sample
test statistic may be calculated as:

\begin{align}
  \hat{Z}_{12}=\frac{\lvert{(K_{1}-\hat\mu_{K_1})-(K_{2}-\hat\mu_{K_2})}\rvert}{\sqrt{\hat\sigma^2_{K_1}+\hat\sigma^2_{K_2}}} 
\end{align}

where \(K_1\), \(K_2\), \(\hat\mu_{K_1}\), \(\hat\mu_{K_2}\),
\(\hat\sigma_{K_1}\), and \(\hat\sigma_{K_2}\) are as defined above.
Estimates of significance of \(\hat{Z}_{12}\) may be obtained from a
standard normal distribution, or permutation. Typically,
\(\hat{Z}_{12}\) is considered a two-tailed test, however directional
(one-tailed) tests may be specified should the empirical situation
require it (36, 38). Simulations reveal that tests based on
\(\hat{Z}_{12}\) have appropriate type I error rates
(\textasciitilde0.05) and reasonable model misspecification rates
(\textasciitilde7-12\%: Supporting Information).

To demonstrate the utility of \(\hat{Z}_{12}\), we compared \(Z_K\) for
two ecologically-relevant traits in plethodontid salamander (Fig. 4):
surface area to volume ratios (SA:V) and relative body width
(\(\frac{BW}{SVL}\)) (41, 42). While both traits contained significant
phylogenetic signal, tests based on \(\hat{Z}_{12}\) revealed that the
degree of phylogenetic signal was significantly stronger in SA:V
(\(\hat{Z}_{12}=4.13\); \(P=0.000036\): Fig. 4). Biologically, this
observation may be interpreted by the fact that the tropical species --
which form a monophyletic group within plethodontids -- display greater
variation in SA:V, which covaries with disparity in their climatic
niches (42). Thus, greater phylogenetic signal in SA:V is to be
expected.

\begin{figure}
\centering
\includegraphics[width=8.25cm]{{fig.4}.png}
\caption{(A) Linear measures for relative body size, and regions of the
body used to estimate surface area to volume (SA:V) ratios. (B)
Permutation distributions of phylogenetic signal for SA:V and
\(\frac{BW}{SVL}\), with observed values shown as vertical bars. (C)
Effect sizes (\(Z_K\)) for SA:V and \(\frac{BW}{SVL}\), with their 95\%
confidence intervals (CI not standardized by \(\sqrt(n)\)).{}}
\end{figure}

\hypertarget{discussion}{%
\section{Discussion}\label{discussion}}

It is common in comparative evolutionary studies to characterize the
phylogenetic signal in phenotypic traits to determine the extent to
which shared evolutionary history has generated trait covariation among
taxa. However, while numerous analytical approaches may be used to
quantify phylogenetic signal (11, 12, 14--16), methods that explicitly
measure the strength of phylogenetic signal, or facilitate comparisons
among datasets, have remained underdeveloped. We evaluated the precision
of one common measure, Pagel's \(\lambda\), and explored its efficacy
for characterizing the strength of phylogenetic signal in phenotypic
data. Using computer simulations, we found that \(\lambda\) behaves as a
Bernoulli random variable, with estimates that are increasingly skewed
at larger and smaller input levels of phylogenetic signal. Further, the
precision of \(\lambda\) in estimating actual levels of phylogenetic
signal varies with both tree size (see also ref. (23)) and input levels
of phylogentetic signal. From these findings we conclude that
\(\lambda\) is not a reliable indicator of the observed strength of
phylogenetic signal in phenoytpic datasets, and should not be used as an
effect size for comparing the degree of phylogenetic signal between
datasets.

As an alternative, we described a standardized effect size (\(Z\)) for
assessing the strength of phylogenetic signal. \(Z\) expresses the
magnitude of phylogenetic signal as a standard normal deviate, which is
easily interpretable as the strength of phylogenetic signal relative to
the mean. We applied this concept to both \(\lambda\) and \emph{K}, and
found that \(Z_K\) was a better estimate of the strength of phylogenetic
signal in phenotypic data. First, values of \(Z_K\) more accurately
tracked known changes in the magnitude of phylogenetic signal, as
demonstrated by the near linear relationship between \(Z_K\) and input
signal. Additionally, the precision of \(Z_K\) was more consistent
across the range of input levels of phylogenetic signal (Fig S1;
Supporting Information). Thus, \(Z_K\) is a more reliable measure of the
relative strength of phylogenetic signal, and places that effect on a
common and comparable scale. We therefore recommend that future studies
interested in evaluating the strength of phylogenetic signal incorporate
\(Z_K\) as a statistical measure of this effect.

Next we proposed a two-sample test (\(\hat{Z}_{12}\)), which provides a
formal statistical procedure for determining whether the strength of
phylogenetic signal is greater in one phenotypic trait as compared to
another. Prior studies have summarized patterns of variation in
phylogenetic signal across datasets using summary test values, such as
\emph{K} (12). However, because \emph{K} does not scale linearly with
input levels of phylogenetic signal (Fig. 2), and its variance increases
with increasing strength of phylogenetic signal (18, 20), it should not
be considered an effect size that measures the strength of phylogenetic
signal on a common scale. By contrast, standardizing \emph{K} to \(Z_K\)
via equation 2 alleviates these concerns, and facilitates formal
statistical comparisons of the strength of signal across datasets. Thus
when viewed from this perspective, the approach developed here aligns
well with other statistical approaches such as meta-analysis (32, 43,
44), where summary statistics across datasets are converted to
standardized effect sizes for subsequent ``higher order'' statistical
summaries or comparisons. As such, our approach enables evolutionary
biologists to quantitatively examine the relative strength of
phylogenetic signal across a wide range of phenotypic traits, and thus
opens the door for future discoveries that inform on how phenotypic
diversity accumulates in macroevolutionary time across the tree of life.

One important advantage of the approach advocated here is that the
resulting effect sizes (\(Z_K\)) are dimensionless, as the units of
measurement cancel out during the calculation of \(Z\) (45). Thus,
\(Z_K\) represents the strength of phylogenetic signal on a common and
comparable scale -- measured in standard deviations -- regardless of the
initial units and original scale of the phenotypic variables under
investigation. This means that the strength of phylogenetic signal may
be compared across datasets for continuous phenotypic traits measured in
different units and scale, because those units have been standardized
through their conversion to \(Z_K\). For example, our approach could be
utilized to determine whether the strength of phylogenetic signal (say,
in response to ecological differentiation) is stronger in morphological
traits (linear traits: \(mm\)), physiological traits (metabolic rate:
\(\frac{O^2}{min}\)), or behavioral traits (aggression:
\(\frac{\#{displays}}{second}\)). In fact, our empirical example
provided just such a comparison, as SA:V is represented in \(mm^{-1}\)
while relative body size is a unitless ratio (\(\frac{BW}{SVL}\)).
Additionally, our method is capable of comparing the strength of
phylogenetic signal in traits of different dimensionality, as estimates
of phylogenetic signal using \emph{K} have been generalized for
multivariate data (16). Furthermore, tests based on \(\hat{Z}_{12}\) may
be utilized for comparing the strength of phylogenetic signal among
datasets containing a different number of variables, and even for
phenotypes obtained from species in different lineages, because their
phylogenetic non-independence and observed variation are taken into
account in the generation of the empirical sampling distribution via
permutation.

This study is not the first to compare \(\lambda\) and \emph{K} for
their ability as statistics to measure phylogenetic signal. Our results
for \(\lambda\) and \emph{K} values are consistent with those found in
the simulations performed by Münkemüller et al.~(18), but that study
investigated type I error rates and statistical power, finding that
\(\lambda\) performed better in both regards, irrespective of species
number in trees. Although not the central focus of their study, the same
tendency for variable \(\lambda\) and consistent \emph{K} at
intermediate phylogenetic signal strengths was observed (Fig. 2 of ref.
(18)). Recent work by Molina-Venegas and Rodríguez (21) found that
\emph{K} but not \(\lambda\) tended to inflate the estimate of
phylogenetic signal, leading to moderate type I and type II biases, if
polytomic chronograms were used. Their work more thoroughly addressed
previous observations of inflated \emph{K} for incompletely resolved
phylogenetic trees (18, 46). An interesting question is whether an
inflated \emph{K} value leads to an inflated \(Z_K\) or does a tendency
of a particular tree to inflate estimates of \emph{K} also inflate the
values in random permutations of a test, in which case \(Z_K\) is robust
to polytomies? We repeated the analyses in Figs. 1 \& 2, adjusting trees
to have 20\% collapsed nodes, per the technique of Molina-Venegas and
Rodríguez (21), and found results were consistent (Supporting
Information). This confirms that any tendency of incompletely resolved
trees to inflate \emph{K} as a descriptive statistic does not inflate
\(Z_K\) as an effect size. Furthermore, because comparison of effect
sizes in a test is a comparison of locations of observed values in their
sampling distributions, which would shift concomitantly because of this
tendency, the \(Z_{12}\) test statistic in equation 4 appears to be
robust in spite of unresolved trees.

Phylogenetic signal can be thought of as both an attribute to be
measured in the data and a parameter that can be tuned to account for
the phylogenetic non-independence among observations, for analysis of
the data. As such, \(\lambda\) is appealing, as a statistic that
potentially fulfills both roles. However, the inability to estimate
phylogenetic signal with \(\lambda\) for data simulated with known
phylogenetic signal is troublesome, and we recommend evolutionary
biologists refrain from viewing it as a statistic to describe the amount
of phylogenetic signal in the data. Interestingly, \emph{K} -- when
standardized to an effect size \(Z_K\) -- is a better statistic for
measuring the amount of phylogenetic signal in data simulated with
respect to known levels of \(\lambda\). Although \(\lambda\) might be
viewed as an important parameter for modifying the the conditional
estimation of linear model coefficients with respect to phylogeny, it is
neither a statistic that has meaningful comparative value as a measure
of phylogenetic signal nor a statistic that lends itself well to
reliable calculation of a test statistic. By contrast, \emph{K} has been
shown here to be a reliable statistic, but only when standardized by the
mean and standard deviation of its empirical sampling distribution
(i.e., when converted to the effect size, \(Z_K\)). Because one has
control over the number of permutations used in analysis, one can be
assured with many permutations that the empirical sampling distribution
is representative of true probability distributions (10). Given the
greater consistency in estimates of \(Z_K\) across tree sizes and input
signal, it is difficult to imagine a hypothesis test that can improve
equation 4 for efficiently comparing phylogenetic signal for different
traits, different trees, or a combination of both.

\hypertarget{methods}{%
\section{Methods}\label{methods}}

\textbf{Simulations.} Simulations were conducted by generating
pure-birth phylogenies at each of six different tree sizes
(\(n=2^5, 2^6, \cdots, 2^{10}\)), and with differing levels of
phylogenetic signal (\(\lambda=0.0, 0.5, \cdots, 1.0\)). We generated 50
random trees for each intersection of tree size and \(\lambda\). For
each \(\lambda\) within each tree size, continuous traits were then
simulated on each phylogeny under a BM model of evolution. For each set
of 50 trees we measured the mean values of \(\hat{\lambda}\) and
\emph{K}, their standard deviation, and calculated the Shapiro-Wilk
\(W\) statistic as a departure from normality (symmetry). For the
latter, a value of \(1.0\) indicates normally distributed values, while
departures from \(1.0\) indicate skewness. Simulations were then
repeated for both balanced and pectinate trees, which yielded
qualitatively similar results (see Supporting Information). Trees
containing polytomies, and an evaluation of \(\hat{\lambda}\) from
models of linear regression and phylogenetic ANOVA, were also
investigated, and results were qualitatively similar to those reported
above (see Supporting Information).

\textbf{Empirical Data.} Surface area to volume ratios (SA:V) and
relative body width (\(\frac{BW}{SVL}\)) measures were obtained from
individuals of 305 species, from which species means were obtained (41,
42). A time-dated molecular phylogeny for the group (47) was pruned to
match the species in the phenotypic dataset. The phylogenetic signal in
each trait was then characterized using \emph{K}, which was converted to
its effect size (\(Z_K\)) using \texttt{geomorph} 3.3.1 (48, 49), and
routines by the authors (\textbf{to be incorporated in \texttt{geomorph}
upon acceptance}).

\showmatmethods
\showacknow
\pnasbreak

\hypertarget{refs}{}
\leavevmode\hypertarget{ref-Felsenstein1985}{}%
1. Felsenstein J (1985) Phylogenies and the comparative method.
\emph{American Naturalist} 125(1):1--15.

\leavevmode\hypertarget{ref-HarveyPagel1991}{}%
2. Harvey PH, Pagel MD (1991) \emph{The comparative method in
evolutionary biology} (Oxford University Press, Oxford).

\leavevmode\hypertarget{ref-Grafen1989}{}%
3. Grafen A (1989) The phylogenetic regression. \emph{Philosophical
Transactions of the Royal Society of London B, Biological Sciences}
326:119--157.

\leavevmode\hypertarget{ref-GarlandIves2000}{}%
4. Garland TJ, Ives AR (2000) Using the past to predict the present:
Confidence intervals for regression equations in phylogenetic
comparative methods. \emph{American Naturalist} 155:346--364.

\leavevmode\hypertarget{ref-Rohlf2001}{}%
5. Rohlf FJ (2001) Comparative methods for the analysis of continuous
variables: Geometric interpretations. \emph{Evolution} 55:2143--2160.

\leavevmode\hypertarget{ref-MartinsHansen1997}{}%
6. Martins EP, Hansen TF (1997) Phylogenies and the comparative method:
A general approach to incorporating phylogenetic information into the
analysis of interspecific data. \emph{American Naturalist} 149:646--667.

\leavevmode\hypertarget{ref-OMeara_et_al2006}{}%
7. O'Meara BC, Ane C, Sanderson MJ, Wainwright PC (2006) Testing for
different rates of continuous trait evolution using likelihood.
\emph{Evolution} 60:922--933.

\leavevmode\hypertarget{ref-Beaulieu_et_al2012}{}%
8. Beaulieu JM, Jhwueng DC, Boettiger C, O'Meara BC (2012) Modeling
stabilizing selection: Expanding the ornstein-uhlenbeck model of
adaptive evolution. \emph{Evolution} 66:2369--2383.

\leavevmode\hypertarget{ref-Adams2014b}{}%
9. Adams DC (2014) A method for assessing phylogenetic least squares
models for shape and other high-dimensional multivariate data.
\emph{Evolution} 68:2675--2688.

\leavevmode\hypertarget{ref-AdamsCollyer2018b}{}%
10. Adams DC, Collyer ML (2018) Phylogenetic anova: Group-clade
aggregation, biological challenges, and a refined permutation procedure.
\emph{Evolution} 72(6):1204--1215.

\leavevmode\hypertarget{ref-Pagel1999}{}%
11. Pagel MD (1999) Inferring the historical patterns of biological
evolution. \emph{Nature} 401:877--884.

\leavevmode\hypertarget{ref-Blomberg_et_al2003}{}%
12. Blomberg SP, Garland T, Ives AR (2003) Testing for phylogenetic
signal in comparative data: Behavioral traits are more labile.
\emph{Evolution} 57:717--745.

\leavevmode\hypertarget{ref-Revell_et_al2008}{}%
13. Revell LJ, Harmon LJ, Collar DC (2008) Phylogenetic signal,
evolutionary process, and rate. \emph{Systematic Biology} 57:591--601.

\leavevmode\hypertarget{ref-Abouheif1999}{}%
14. Abouheif E (1999) A method for testing the assumption of
phylogenetic independence in comparative data. \emph{Evolutionary
Ecology Research} 1:895--909.

\leavevmode\hypertarget{ref-Gittleman1990}{}%
15. Gittleman JL, Kot M (1990) Adaptation: Statistics and a null model
for estimating phylogenetic effects. \emph{Systematic Zoology}
39(3):227--241.

\leavevmode\hypertarget{ref-Adams2014a}{}%
16. Adams DC (2014) A generalized Kappa statistic for estimating
phylogenetic signal from shape and other high-dimensional dultivariate
data. \emph{Systematic Biology} 63:685--697.

\leavevmode\hypertarget{ref-Klingenberg2010}{}%
17. Klingenberg CP, Gidaszewski NA (2010) Testing and quantifying
phylogenetic signals and homoplasy in morphometric data.
\emph{Systematic biology} 59(3):245--261.

\leavevmode\hypertarget{ref-Munkemuller_et_al2012}{}%
18. Münkemüller T, et al. (2012) How to measure and test phylogenetic
signal. \emph{Methods in Ecology and Evolution} 3:743--756.

\leavevmode\hypertarget{ref-Pavoine2012}{}%
19. Pavoine S, Ricotta C (2012) Testing for phylogenetic signal in
biological traits: The ubiquity of cross-product statistics.
\emph{Evolution: International Journal of Organic Evolution}
67(3):828--840.

\leavevmode\hypertarget{ref-DinizFilho2012}{}%
20. Diniz-Filho JAF, Santos T, Rangel TF, Bini LM (2012) A comparison of
metrics for estimating phylogenetic signal under alternative
evolutionary models. \emph{Genetics and Molecular Biology}
35(3):673--679.

\leavevmode\hypertarget{ref-MolinaVenegas2017}{}%
21. Molina-Venegas R, Rodríguez MA (2017) Revisiting phylogenetic
signal; strong or negligible impacts of polytomies and branch length
information? \emph{BMC evolutionary biology} 17(1):53.

\leavevmode\hypertarget{ref-Revell2010}{}%
22. Revell LJ (2010) Phylogenetic signal and linear regression on
species data. \emph{Methods in Ecology and Evolution} 1:319--329.

\leavevmode\hypertarget{ref-Boettiger_et_al2012}{}%
23. Boettiger C, Coop G, Ralph P (2012) Is your phylogeny informative?
Measuring the power of comparative methods. \emph{Evolution}
67:2240--2251.

\leavevmode\hypertarget{ref-Freckleton_et_al2002}{}%
24. Freckleton RP, Harvey PH, Pagel M (2002) Phylogenetic analysis and
comparative data: A test and review of evidence. \emph{American
Naturalist} 160:712--726.

\leavevmode\hypertarget{ref-Cooper2010}{}%
25. Cooper N, Jetz W, Freckleton RP (2010) Phylogenetic comparative
approaches for studying niche conservatism. \emph{Journal of
Evolutionary Biology} 23(12):2529--2539.

\leavevmode\hypertarget{ref-Bose2019}{}%
26. Bose R, Ramesh BR, Pélissier R, Munoz F (2019) Phylogenetic
diversity in the western ghats biodiversity hotspot reflects
environmental filtering and past niche diversification of trees.
\emph{Journal of Biogeography} 46(1):145--157.

\leavevmode\hypertarget{ref-Vandelook2019}{}%
27. Vandelook F, et al. (2019) Nectar traits differ between pollination
syndromes in balsaminaceae. \emph{Annals of Botany} 124(2):269--279.

\leavevmode\hypertarget{ref-DeMeester2019}{}%
28. De Meester G, Huyghe K, Van Damme R (2019) Brain size, ecology and
sociality: A reptilian perspective. \emph{Biological Journal of the
Linnean Society} 126(3):381--391.

\leavevmode\hypertarget{ref-Pintanel2019}{}%
29. Pintanel P, Tejedo M, Ron SR, Llorente GA, Merino-Viteri A (2019)
Elevational and microclimatic drivers of thermal tolerance in andean
pristimantis frogs. \emph{Journal of Biogeography} 46(8):1664--1675.

\leavevmode\hypertarget{ref-Su2019}{}%
30. Su G, Villéger S, Brosse S (2019) Morphological diversity of
freshwater fishes differs between realms, but morphologically extreme
species are widespread. \emph{Global ecology and biogeography}
28(2):211--221.

\leavevmode\hypertarget{ref-Forbes2011}{}%
31. Forbes C, Evans M, Hastings N, Peacock B (2011) \emph{Statistical
distributions} (John Wiley \& Sons).

\leavevmode\hypertarget{ref-Glass1976}{}%
32. Glass GV (1976) Primary, secondary, and meta-analysis of research.
\emph{Educational Researcher} 5:3--8.

\leavevmode\hypertarget{ref-Cohen1988}{}%
33. Cohen J (1988) \emph{Statistical power analysis for the behavioral
sciences} (Routledge).

\leavevmode\hypertarget{ref-Rosenthal1994}{}%
34. Rosenthal R (1994) The handbook of research synthesis. ed Cooper LV
H Hedges (Russell Sage Foundation), pp 231--244.

\leavevmode\hypertarget{ref-Collyer_et_al2015a}{}%
35. Collyer ML, Sekora DJ, Adams DC (2015) A method for analysis of
phenotypic change for phenotypes described by high-dimensional data.
\emph{Heredity} 115:357--365.

\leavevmode\hypertarget{ref-AdamsCollyer2016}{}%
36. Adams DC, Collyer ML (2016) On the comparison of the strength of
morphological integration across morphometric datasets. \emph{Evolution}
70:2623--2631.

\leavevmode\hypertarget{ref-CollyerAdams2018}{}%
37. Collyer ML, Adams DC (2018) RRPP: An r package for fitting linear
models to high-dimensional data using residual randomization.
\emph{Methods in Ecology and Evolution} 9:1772--1779.

\leavevmode\hypertarget{ref-AdamsCollyer2019b}{}%
38. Adams DC, Collyer ML (2019) Comparing the strength of modular
signal, and evaluating alternative modular hypotheses, using covariance
ratio effect sizes with morphometric data. \emph{Evolution}
73(12):2352--2367.

\leavevmode\hypertarget{ref-AdamsCollyer2019}{}%
39. Adams DC, Collyer ML (2019) Phylogenetic comparative methods and the
evolution of multivariate phenotypes. \emph{Annual Review of Ecology,
Evolution, and Systematics} 50:405--425.

\leavevmode\hypertarget{ref-Orme2013}{}%
40. Orme D, et al. (2013) CAPER: Comparative analyses of phylogenetics
and evolution in r. \emph{Methods in Ecology and Evolution} 3:145--151.

\leavevmode\hypertarget{ref-Baken2019}{}%
41. Baken EK, Adams DC (2019) Macroevolution of arboreality in
salamanders. \emph{Ecology and Evolution} 9(12):7005--7016.

\leavevmode\hypertarget{ref-Baken2020}{}%
42. Baken EK, Mellenthin LE, Adams DC (2020) Macroevolution of
desiccation‐related morphology in plethodontid salamanders as inferred
from a novel surface area to volume ratio estimation approach.
\emph{Evolution} 74:476--486.

\leavevmode\hypertarget{ref-HedgesOlkin1985}{}%
43. Hedges L. V., Olkin I (1985) \emph{Statistical methods for
meta-analysis} (Elsevier).

\leavevmode\hypertarget{ref-Arnqvist1995}{}%
44. Arnqvist G., Wooster D (1995) Meta-analysis: Synthesizing research
findings in ecology and evolution. \emph{Trends in Ecology and
Evolution} 10:236--240.

\leavevmode\hypertarget{ref-SokalRohlf2012}{}%
45. Sokal R. R., Rohlf FJ (2012) \emph{Biometry} (W.H. Freeman \& Co.,
San Francisco). 4th Ed.

\leavevmode\hypertarget{ref-Davies2012}{}%
46. Davies TJ, Kraft NJ, Salamin N, Wolkovich EM (2012) Incompletely
resolved phylogenetic trees inflate estimates of phylogenetic
conservatism. \emph{Ecology} 93(2):242--247.

\leavevmode\hypertarget{ref-Bonett2017}{}%
47. Bonett RM, Blair AL (2017) Evidence for complex life cycle
constraints on salamander body form diversification. \emph{Proceedings
of the National Academy of Sciences, USA} 114:9936--9941.

\leavevmode\hypertarget{ref-AdamsOtarola2013}{}%
48. Adams DC, Otárola-Castillo E (2013) Geomorph: An r package for the
collection and analysis of geometric morphometric shape data.
\emph{Methods in Ecology and Evolution} 4:393--399.

\leavevmode\hypertarget{ref-AdamsGeomorph}{}%
49. Adams DC, Collyer ML, Kaliontzopoulou A (2020) Geomorph: Software
for geometric morphometric analyses. R package version 3.3.1. Available
at: \url{https://cran.r-project.org/package=geomorph}.



% Bibliography
% \bibliography{pnas-sample}

\end{document}

