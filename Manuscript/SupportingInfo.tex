% Options for packages loaded elsewhere
\PassOptionsToPackage{unicode}{hyperref}
\PassOptionsToPackage{hyphens}{url}
%
\documentclass[
]{article}
\usepackage{lmodern}
\usepackage{amssymb,amsmath}
\usepackage{ifxetex,ifluatex}
\ifnum 0\ifxetex 1\fi\ifluatex 1\fi=0 % if pdftex
  \usepackage[T1]{fontenc}
  \usepackage[utf8]{inputenc}
  \usepackage{textcomp} % provide euro and other symbols
\else % if luatex or xetex
  \usepackage{unicode-math}
  \defaultfontfeatures{Scale=MatchLowercase}
  \defaultfontfeatures[\rmfamily]{Ligatures=TeX,Scale=1}
\fi
% Use upquote if available, for straight quotes in verbatim environments
\IfFileExists{upquote.sty}{\usepackage{upquote}}{}
\IfFileExists{microtype.sty}{% use microtype if available
  \usepackage[]{microtype}
  \UseMicrotypeSet[protrusion]{basicmath} % disable protrusion for tt fonts
}{}
\makeatletter
\@ifundefined{KOMAClassName}{% if non-KOMA class
  \IfFileExists{parskip.sty}{%
    \usepackage{parskip}
  }{% else
    \setlength{\parindent}{0pt}
    \setlength{\parskip}{6pt plus 2pt minus 1pt}}
}{% if KOMA class
  \KOMAoptions{parskip=half}}
\makeatother
\usepackage{xcolor}
\IfFileExists{xurl.sty}{\usepackage{xurl}}{} % add URL line breaks if available
\IfFileExists{bookmark.sty}{\usepackage{bookmark}}{\usepackage{hyperref}}
\hypersetup{
  pdftitle={Supporting Information for: A Standardized Effect Size for Evaluating and Comparing the Strength of Phylogenetic Signal},
  pdfauthor={Dean C. Adams, Erica K. Baken, and Michael L. Collyer},
  hidelinks,
  pdfcreator={LaTeX via pandoc}}
\urlstyle{same} % disable monospaced font for URLs
\usepackage[margin=1in]{geometry}
\usepackage{graphicx,grffile}
\makeatletter
\def\maxwidth{\ifdim\Gin@nat@width>\linewidth\linewidth\else\Gin@nat@width\fi}
\def\maxheight{\ifdim\Gin@nat@height>\textheight\textheight\else\Gin@nat@height\fi}
\makeatother
% Scale images if necessary, so that they will not overflow the page
% margins by default, and it is still possible to overwrite the defaults
% using explicit options in \includegraphics[width, height, ...]{}
\setkeys{Gin}{width=\maxwidth,height=\maxheight,keepaspectratio}
% Set default figure placement to htbp
\makeatletter
\def\fps@figure{htbp}
\makeatother
\setlength{\emergencystretch}{3em} % prevent overfull lines
\providecommand{\tightlist}{%
  \setlength{\itemsep}{0pt}\setlength{\parskip}{0pt}}
\setcounter{secnumdepth}{-\maxdimen} % remove section numbering

\title{Supporting Information for: A Standardized Effect Size for Evaluating
and Comparing the Strength of Phylogenetic Signal}
\author{Dean C. Adams, Erica K. Baken, and Michael L. Collyer}
\date{}

\begin{document}
\maketitle

Here we provide additional supporting information referenced in the main
document, which include additional analyses, and simulation results
across a wider set of input conditions. As before, simulations were
conducted on six different tree sizes (\(n=2^5, 2^6, \cdots, 2^{10}\)),
and with differing levels of phylogenetic signal
(\(\lambda=0.0, 0.5, \cdots, 1.0\)). We generated 100 random trees for
each intersection of tree size and \(\lambda\). For each \(\lambda\)
within each tree size, continuous traits were then simulated on each
phylogeny under a BM model of evolution. For each set of 100 trees we
measured the mean values of \(\hat{\lambda}\) and \(\kappa\), their
standard deviation, and calculated the Shapiro-Wilk \(W\) statistic as a
departure from normality (symmetry). For the latter, a value of \(1.0\)
indicates normally distributed values, while departures from \(1.0\)
indicate skewness.

\hypertarget{simulations-on-pectinate-phylogenies}{%
\subsection{Simulations on Pectinate
Phylogenies}\label{simulations-on-pectinate-phylogenies}}

Results from simulations on pectinate phylogenies largely mirrored those
found on pure-birth trees. For \(\hat{\lambda}\), the mean value
increased with increasing input signal, but was negatively biased, and
was less than the input value across most of its range (Fig. S1: black
line). Additionally, the precision of \(\hat{\lambda}\) varied with
differing input levels, with the greatest variation found at
intermediate values of \(\lambda\) (Fig. S1 red line). Finally, the
distribution of \(\hat{\lambda}\) was not normal, and became more skewed
at more extreme values of \(\lambda\) (Fig. S1 blue line). \hfill\break

For \(\hat\kappa\), mean values increased with increasing phylogenetic
signal, though as was found with pure-birth trees, the increase was
nonlinear (Fig. S2 black line). Likewise, variation increased with
increasing phylogenetic signal (Fig. S2 red line), though the
distribution of \(\hat\kappa\) was more normally distributed throughout
its range, and across different tree sizes, as compared with
\(\hat{\lambda}\), thought there was some slight skew for high input
levels of phylogenetic signal on large phylogenies (Fig. S2 blue line).
\hfill\break

With respect to effect sizes, both \(Z_{\lambda}\) and \(Z_{\kappa}\)
increased with increasing input phylogenetic signal, but
\(\hat{\lambda}\) was strongly affected by tree size, whereas
\(Z_{\kappa}\) was more consistent (Fig. S3). Also, \(Z_{\kappa}\)
increased more linearly with increasing levels of phylogenetic signal,
and its standard deviation across input signal was more even across tree
sizes, implying more consistent precision.

\includegraphics[width=0.95\linewidth]{fig.S1}

\textbf{Figure S1}. Response of Pagel's \(\lambda\) to increasing
strength of Brownian motion on pectinate trees. Gray line signifies the
1:1 line where the input value matches the estimate.

\includegraphics[width=0.95\linewidth]{fig.S2}

\textbf{Figure S2}. Response of Blomberg's \(\kappa\) to increasing
strength of Brownian motion on pectinate trees. Gray line signifies the
1:1 line where the input value matches the estimate.

\includegraphics[width=0.95\linewidth]{fig.S3}

\textbf{Figure S3}. Response of effect sizes \(Z_{\lambda}\) and
\(Z_{\kappa}\) to increasing strength of Brownian motion.

\newpage

\hypertarget{simulations-on-balanced-phylogenies}{%
\subsection{Simulations on Balanced
Phylogenies}\label{simulations-on-balanced-phylogenies}}

Results from simulations on balanced phylogenies also mirrored those
found on pure-birth trees. For \(\hat{\lambda}\), the mean value
increased with increasing input signal, but was negatively biased, and
was less than the input value across most of its range (Fig. S4: black
line). Additionally, the precision of \(\hat{\lambda}\) varied with
differing input levels, with the greatest variation found at
intermediate values of \(\lambda\) (Fig. S4 red line). Finally, the
distribution of \(\hat{\lambda}\) was not normal, and became more skewed
at more extreme values of \(\lambda\) (Fig. S4 blue line). \hfill\break

For \(\hat\kappa\), mean values increased with increasing phylogenetic
signal, though as was found with pure-birth trees, the increase was
nonlinear (Fig. S5 black line). Likewise, variation increased with
increasing phylogenetic signal (Fig. S5 red line), though the
distribution of \(\hat\kappa\) was more normally distributed throughout
its range, and across different tree sizes, as compared with
\(\hat{\lambda}\), thought there was some slight skew for high input
levels of phylogenetic signal on large phylogenies (Fig. S5 blue line).
\hfill\break

With respect to effect sizes, both \(Z_{\lambda}\) and \(Z_{\kappa}\)
increased with increasing input phylogenetic signal, but
\(\hat{\lambda}\) was strongly affected by tree size (fig.~S6). Also,
\(Z_{\kappa}\) increased more linearly with increasing levels of
phylogenetic signal, though there was some affect of tree size with
balanced phylogenies. But as before, its standard deviation across input
signal was more even across tree sizes, implying more consistent
precision.

\includegraphics[width=0.95\linewidth]{fig.S4}

\textbf{Figure S4}. Response of Pagel's \(\lambda\) to increasing
strength of Brownian motion on balanced trees. Gray line signifies the
1:1 line where the input value matches the estimate.

\includegraphics[width=0.95\linewidth]{fig.S5}

\textbf{Figure S5}. Response of Blomberg's \(\kappa\) to increasing
strength of Brownian motion on balanced trees. Gray line signifies the
1:1 line where the input value matches the estimate.

\includegraphics[width=0.95\linewidth]{fig.S6}

\textbf{Figure S6}. Response of effect sizes \(Z_{\lambda}\) and
\(Z_{\kappa}\) to increasing strength of Brownian motion.

\newpage

\hypertarget{simulations-on-phylogenies-containing-polytomies}{%
\subsection{Simulations on Phylogenies Containing
Polytomies}\label{simulations-on-phylogenies-containing-polytomies}}

We also investigaged the effect of unresolved phylogenies on estimates
of phylogenetic signal by adjusting pure-birth trees to have 50\%
collapsed nodes, following the procedures of (1). Results from
simulations on phylogenies containing polytomies mirrored those found on
fully resolved trees. For \(\hat{\lambda}\), the mean value increased
with increasing input signal, but was negatively biased, and was less
than the input value across most of its range (Fig. S7: black line).
Additionally, the precision of \(\hat{\lambda}\) varied with differing
input levels, with the greatest variation found at intermediate values
of \(\lambda\) (Fig. S7 red line). Finally, the distribution of
\(\hat{\lambda}\) was not normal, and became more skewed at more extreme
values of \(\lambda\) (Fig. S7 blue line). \hfill\break

For \(\hat\kappa\), mean values increased with increasing phylogenetic
signal, though as was found with pure-birth trees, the increase was
nonlinear (Fig. S8 black line). Likewise, variation increased with
increasing phylogenetic signal (Fig. S8 red line), though the
distribution of \(\hat\kappa\) was more normally distributed throughout
its range, and across different tree sizes, as compared with
\(\hat{\lambda}\) (Fig. S8 blue line). \hfill\break

With respect to effect sizes, both \(Z_{\lambda}\) and \(Z_{\kappa}\)
increased with increasing input phylogenetic signal, but
\(\hat{\lambda}\) was strongly affected by tree size, whereas
\(Z_{\kappa}\) was more consistent (Fig. S9). Also, \(Z_{\kappa}\)
increased more linearly with increasing levels of phylogenetic signal,
and its standard deviation across input signal was more even across tree
sizes, implying more consistent precision. Thus, polytomies do not exert
an appreciable effect on \(Z_{\kappa}\).

\includegraphics[width=0.95\linewidth]{fig.S7}

\textbf{Figure S7}. Response of Pagel's \(\lambda\) to increasing
strength of Brownian motion on pure-birth trees containing polytomies.
Gray line signifies the 1:1 line where the input value matches the
estimate.

\includegraphics[width=0.95\linewidth]{fig.S8}

\textbf{Figure S8}. Response of Blomberg's \(\kappa\) to increasing
strength of Brownian motion on pure-birth trees containing polytomies.
Gray line signifies the 1:1 line where the input value matches the
estimate.

\includegraphics[width=0.95\linewidth]{fig.S9}

\textbf{Figure S9}. Response of effect sizes \(Z_{\lambda}\) and
\(Z_{\kappa}\) to increasing strength of Brownian motion.

\newpage

\hypertarget{simulations-of-phylogenetic-regression-and-anova}{%
\subsection{Simulations of Phylogenetic Regression and
ANOVA}\label{simulations-of-phylogenetic-regression-and-anova}}

We also investigated the effect of incorporating \(\lambda\) in PGLS
analyses (phylogenetic regression and ANOVA). Here, patterns for
\(\hat\lambda\), \(\hat\kappa\), \(Z_\lambda\) and \(Z_\kappa\) were
virtually identical to those found previously. In essence,
\(\hat\lambda\) was not suitable as a test statistic representing
phylogenetic signal due to all properties previously shown (Figs. S10 \&
S11), \(\hat\kappa\) was more appropriate, but associated non-linearly
with signal strength (Figs. S12 \& 13), while \(Z_\kappa\) was found to
be a superior effect size as compared with \(Z_\lambda\) (Figs S14 \&
S15). \hfill\break

With respect to model performance, input parameters (\(\beta\)) were
well estimated in phylogenetic regression and ANOVA when \(\lambda\) was
incorporated (Figs S16 \& S17), implying that \(\lambda\) may be used as
a tuning parameter for these models. Additionally, type I error and
statistical power were unaffected by the inclusion of \(\lambda\) (Figs
S18 \& S19). This latter result confirms earlier findings of (2) for
phylogenetic regression, and extends them to the case of phylogenetic
ANOVA.

\includegraphics[width=0.95\linewidth]{fig.S10}

\textbf{Figure S10}. Response of Pagel's \(\lambda\) to increasing
strength of Brownian motion in phylogenetic regression (incorporating
\(\lambda\)). Gray line signifies the 1:1 line where the input value
matches the estimate.

\includegraphics[width=0.95\linewidth]{fig.S11}

\textbf{Figure S11}. Response of Pagel's \(\lambda\) to increasing
strength of Brownian motion in phylogenetic ANOVA (incorporating
\(\lambda\)). Gray line signifies the 1:1 line where the input value
matches the estimate.

\includegraphics[width=0.95\linewidth]{fig.S12}

\textbf{Figure S12}. Response of Blomberg's \(\kappa\) to increasing
strength of Brownian motion in phylogenetic regression (incorporating
\(\lambda\)). Gray line signifies the 1:1 line where the input value
matches the estimate.

\includegraphics[width=0.95\linewidth]{fig.S13}

\textbf{Figure S13}. Response of Blomberg's \(\kappa\) to increasing
strength of Brownian motion in phylogenetic ANOVA (incorporating
\(\lambda\)). Gray line signifies the 1:1 line where the input value
matches the estimate.

\includegraphics[width=0.95\linewidth]{fig.S14}

\textbf{Figure S14}. Response of effect sizes \(Z_{\lambda}\) and
\(Z_{\kappa}\) to increasing strength of Brownian motion in phylogenetic
regression (incorporating \(\lambda\)).

\includegraphics[width=0.95\linewidth]{fig.S15}

\textbf{Figure S15}. Response of effect sizes \(Z_{\lambda}\) and
\(Z_{\kappa}\) to increasing strength of Brownian motion in phylogenetic
ANOVA (incorporating \(\lambda\)).

\includegraphics[width=0.95\linewidth]{fig.S16}

\textbf{Figure S16}. Parameter estimates from phylogenetic regression
(incorporating \(\lambda\)). Gray line is the 1:1 line and red line is
the estimate.

\includegraphics[width=0.95\linewidth]{fig.S17}

\textbf{Figure S17}. Parameter estimates from phylogenetic ANOVA
(incorporating \(\lambda\)). Gray line is the 1:1 line and red line is
the estimate.

\newpage

\hypertarget{references}{%
\subsection*{References}\label{references}}
\addcontentsline{toc}{subsection}{References}

\hypertarget{refs}{}
\leavevmode\hypertarget{ref-MolinaVenegas2017}{}%
1. Molina-Venegas R, Rodríguez MA (2017) Revisiting phylogenetic signal;
strong or negligible impacts of polytomies and branch length
information? \emph{BMC evolutionary biology} 17(1):53.

\leavevmode\hypertarget{ref-Revell2010}{}%
2. Revell LJ (2010) Phylogenetic signal and linear regression on species
data. \emph{Methods in Ecology and Evolution} 1:319--329.

\end{document}
